\documentclass{article}

\usepackage[T2A]{fontenc}
\usepackage[utf8x]{inputenc}
\usepackage[english,russian]{babel}

% Wide page
\usepackage{fullpage}

% Maths
\usepackage{mathtools}
\mathtoolsset{showonlyrefs} % turn off numbering of unreferenced equations
\usepackage{amsmath,amssymb,amsthm}

% Advanced enumerate
\usepackage{enumitem}

% Multiple columns
\usepackage{multicol}
\setlength\columnsep{-3em} % default columnsep for all pages

% Tables
\usepackage{diagbox} % Diagonal line in table header
\renewcommand{\arraystretch}{1.2} % More space between table rows

% Redefine some useful commands
\renewcommand{\epsilon}{\varepsilon} % nice-looking epsilon

% My operators
\newcommand{\R}{\mathbb{R}}
\renewcommand{\S}{\mathbb{S}}
\DeclareMathOperator{\Exists}{\exists}
\DeclareMathOperator{\Forall}{\forall}
\DeclareMathOperator*{\Tr}{Tr}
\DeclareMathOperator*{\Det}{Det}
\DeclareMathOperator*{\Diag}{Diag}
\DeclareMathOperator*{\Rank}{Rank}
\DeclareMathOperator*{\Dom}{Dom}

\title{\vspace{-2em}A Neural Algorithm of Artistic Style}
\date{}

\begin{document}

\maketitle

Сверточные Глубинные нейронные сети являются наиболее мощной моделью в обработке и анализе картинок. Сверточные нейроные сети состоят из слоев выполняющих небольшие вычисления обрабатывающие визуальную информацию иерархически от первого слоя к последнему. Каждый слой может представляет из себя набор фильтров, каждый из которых достает конкретную фичу из входящего изображения. Таким образом выход каждого слоя состоит и мапа фич -- поразному отфильтрованных выриантов входного изображания.\\



Когда сверточная нейронная сейть обучается на распознование изображений, она собрает некоторое представление изоражения, которое делает информацию об объекте более и более явной в иерархоческом процессе обработки. Поэтому, в течении иерархоческого процесса обучения нейронной сети, входящее изображение трансформируется в некоторое представление которое в первую очередь хранит в себе "содержание" изображения, а не детальное значение пикселей. Мы  можем напрямую визуализировать трансформированное изображения с каждого уровня восстановив изображение только с мапа фич с данного слоя(Fig1 content reconstruction). Высшие слои сети извлекают из изображения высокоуровневую -- более обобщенную информацию о содержании(контенте) изображения, о расположении объектов на входном изображении, практичкески без какой либо привязки к конкретным значениеям пикселей на исходном изображении. В отличие от изображения восстновленного с нижних слоев в которых не сложно воспроизвести практически точноные значения пикселейисходного изображения.\\


Что бы получить представление о стиле исходного изображения, надо использовать признаковое пространство, изначально предназначенное для сбора информации про текстуру изображения. Данное пространство построенно на выходах фильтров с каждого слоя сети и состоит из корелляции между различными выходами фильтров . Включая корелляцию между слоями мы получаем cтационарное, многомасштабное представление входного изображения, захватывающего текстуры.\\


<Пичка с сетью>


Мы можем визуализировать информаицию полученную из этого признакового пространства текстуры изображения построенного с разных слоев сети путем построения изображения, которое соответствует стилю конкретного входного. Действительно, реконструкция стиля производит новую версию исходного изображения которое захватывает основные особенности стиля с точки зрения структур и цвета.


Основной вывод данной статьи заключается в том, что представление о содержании и стиле изображения в сверточной нейронной сети разделимо. То есть мым ожем манипулировать обоими предсталениями независимо для получения новых изображений. Изображения получаются с помощью поиска картинки одновременно соответствующей по содержанию первому изображению и по стилю второму.


Анализ третьей пикчи


Методы построения:


Все приведенные в статье результаты было получены с использованием WGG-Network. Признаковое пространство было получено с помощью 16 срерточных слое и 5 пулингов данной сети, состоящей из 19 слоев. Полносвязная часть сети не использовалась. Max пулинг был заменен на avarage пулинг для улучшения градиентного спуска и получения более явного результата.


Каждый слой сети представляет из себя набор нелинейных фильтров сложность которых увеличивается вместе с номером слоя сети.



Пусть  входное изображение $x$ было получено пропусканием через слои сети. Слой l с $N_l$ различных фильтров имеет  $N_l$  мапов фичей каждый размера  $M_l$ значит отвеы слоя l могут быть сохранены в матрице $ F \in R^{N_l x M_l} $, где $F_{ij}^{l}$  выход i-ого фильтра на j-ом слое.


Для визуализации информации с каждого слоя мы будем запускать градиентный спуск на белом шуме что бы найти изображение которому соответствуют такие же выходы слоев(мары фич)


Пусть $p, x$ оригинальная картинка и сгенеренная, тогда $P^l, F^l$ их представления на l слое тогда определим квадратичную ошибку между двумя представлениями и ее производную с помощью которого градиент будет сходиться к исходному изображению $x$ использую квадртичную ошибку. Пример на первой пикче


При построении пространства представлений на каждом уровне мы находим корелляцию между различными выходами фильтров с помощью постоения матрицы Грама $G^l \in R^{N_l x N_l}$, где $G_{ij}^{l}$ -- кореллляция между выкторизированными мапами фич i и j на слое l

$$
G_{ij}^l = \sum_k F_{ik}^l F_{jk}^l
$$

Для генерации ткстуры которорая соответствует стилю исходного изображения мы будем использовать градиентный спуск с белого шума для поиска другого изображения которое имеет такой же стиль. Будем делать это с помощью минимизвции среднеквадратичного откланения матрицы грама от оригинального. Пусть $a, x$  оригинальное изображение и белый шум, а $A^l, G^l$ представления на слое $l$. Тогда определим ошибку для конкретного слоя и общую ошибку, где $w$ это вес вложения данного слоя в исходную ошибку. определим проихзводную.


Для генерации изображения которое соответствует контенту одного изображению и стилю другого мы будем минимизировать расстояния белого шума к певой картинке по контенту и к второй картинку по стилю



\end{document}
