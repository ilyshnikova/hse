\documentclass{article}

\usepackage[T2A]{fontenc}
\usepackage[utf8x]{inputenc}
\usepackage[english,russian]{babel}

% Wide page
\usepackage{fullpage}

% Maths
\usepackage{mathtools}
\mathtoolsset{showonlyrefs} % turn off numbering of unreferenced equations
\usepackage{amsmath,amssymb,amsthm}

% Advanced enumerate
\usepackage{enumitem}

% Multiple columns
\usepackage{multicol}
\setlength\columnsep{-3em} % default columnsep for all pages

% Tables
\usepackage{diagbox} % Diagonal line in table header
\renewcommand{\arraystretch}{1.2} % More space between table rows

% Redefine some useful commands
\renewcommand{\epsilon}{\varepsilon} % nice-looking epsilon

% My operators
\newcommand{\R}{\mathbb{R}}
\renewcommand{\S}{\mathbb{S}}
\DeclareMathOperator{\Exists}{\exists}
\DeclareMathOperator{\Forall}{\forall}
\newcommand{\Tr}{\mathrm{Tr}}
\newcommand{\Det}{\mathrm{Det}}
\newcommand{\Diag}{\mathrm{Diag}}
\newcommand{\Rank}{\mathrm{Rank}}

\title{\vspace{-2em}Методы оптимизации (ФКН ВШЭ, 2017). Домашняя работа 1.\\Тема: Скорости сходимости и матричные вычисления.}
\author{Срок сдачи: 17 января 2017 (на семинаре)}
\date{}

\begin{document}

\maketitle

\begin{enumerate}[label=\textbf{\arabic*}, leftmargin=0em]
\item Классифицируйте каждую из следующих последовательностей $(r_k)_{k \geq 1}$ по скорости сходимости (линейная/сублинейная/сверхлинейная). Для сверхлинейно сходящихся последовательностей необходимо дополнительно выяснить, имеет ли место квадратичная сходимость.
\begin{multicols}{3}
	\begin{enumerate}
		\item $r_k := (0.99)^k$
		\item $r_k := (0.99)^{k^2}$
		\item $r_k := (0.99)^{2^k}$
		\item $r_k := 1/k$
		\item $r_k := 1/\sqrt{k}$
		\item $r_k := 1/k^2$
		\item $r_k := 1/k!$
		\item $r_k := 1/k^k$
		\item $r_k := \begin{cases}
		\left(0.99\right)^{2^k}, & \text{если $k$ четное}, \\
		\frac{r_{k-1}}{k}, & \text{иначе}
		\end{cases}$
		\item $(r_k) := \left(1, \frac{1}{4}, \frac{1}{2}, \frac{1}{8}, \frac{1}{4}, \frac{1}{16}, \frac{1}{8}, \dots \right)$
	\end{enumerate}
\end{multicols}

\item Рассмотрим следующие три семейства последовательностей:
\begin{enumerate}[ref=(\alph*)]
	\item (Сублинейные) $r_k := C / k^{\gamma}$.\hfill$[C, \gamma > 0]$
	\item (Линейные) $r_k := C q^k$.\hfill$[C > 0, \, q \in (0, 1)]$
	\item (Квадратичные) $r_k := C (C^{-1} R)^{2^k}$.\hfill$[C > 0, \, C^{-1} R \in (0, 1)]$
\end{enumerate}
Обозначим $k(\epsilon) := \min\{ k \geq 1 : r_k \leq \epsilon C \}$~--- необходимое число шагов для достижения заданной относительной точности $\epsilon \in (0, 1)$.

Для каждого из указанных семейств выпишите явную формулу для $k(\epsilon)$. Проанализируйте, насколько сильно $k(\cdot)$ зависит от требуемой точности $\epsilon$ и соответствующего параметра семейства ($\gamma$, $q$, $R$). Заполните следующие таблицы, вписав в пустые ячейки соответствующие числовые значения $k(\epsilon)$:
\begin{multicols}{3}
	
	\begin{tabular}{|l|l|l|l|}\hline
		\multicolumn{4}{|c|}{Сублинейные} \\ \hline
		\multicolumn{1}{|c|}{\diagbox{$\epsilon$}{$\gamma$}} & \multicolumn{1}{c|}{1} & \multicolumn{1}{c|}{2} & \multicolumn{1}{c|}{0.5} \\ \hline
		$10^{-1}$ & \hphantom{$1000$} & \hphantom{$1000$} & \hphantom{$1000$} \\ \hline
		$10^{-3}$ & & & \\ \hline
		$10^{-5}$ & & & \\ \hline
		$10^{-7}$ & & & \\ \hline
		$10^{-12}$ & & & \\ \hline
	\end{tabular}
	\columnbreak
	
	\hspace{-0.5em}\begin{tabular}{|l|l|l|l|}\hline
		\multicolumn{4}{|c|}{Линейные} \\ \hline
		\multicolumn{1}{|c|}{\diagbox{$\epsilon$}{$q$}} & \multicolumn{1}{c|}{0.9} & \multicolumn{1}{c|}{0.999} & \multicolumn{1}{c|}{0.99999} \\ \hline
		$10^{-1}$ & \hphantom{$1000$} & \hphantom{$1000$} & \hphantom{$1000$} \\ \hline
		$10^{-3}$ & & & \\ \hline
		$10^{-5}$ & & & \\ \hline
		$10^{-7}$ & & & \\ \hline
		$10^{-12}$ & & & \\ \hline
	\end{tabular}
	\columnbreak
	
	\begin{flushright}
		\begin{tabular}{|l|l|l|l|} \hline
			\multicolumn{4}{|c|}{Квадратичные} \\ \hline
			\multicolumn{1}{|c|}{\diagbox{$\epsilon$}{$R$}} & \multicolumn{1}{c|}{$0.9 \, C$} & \multicolumn{1}{c|}{$0.999 \, C$} & \multicolumn{1}{c|}{$0.99999 \, C$} \\ \hline
			$10^{-1}$ & \hphantom{$1000$} & \hphantom{$1000$} & \hphantom{$1000$} \\ \hline
			$10^{-3}$ & & & \\ \hline
			$10^{-5}$ & & & \\ \hline
			$10^{-7}$ & & & \\ \hline
			$10^{-12}$ & & & \\ \hline
		\end{tabular}
	\end{flushright}
	
\end{multicols}

\textbf{Рекомендация:} Напишите скрипт, который заполнит все таблицы автоматически. Достаточно выписать одну значимую цифру и показатель мантиссы (например: $3 \times 10^8$)

\item Упростите каждое из из следующих выражений:
\begin{enumerate}
\item $\Tr[(A X B)^{-1} A C B]$\hfill[$A, B, C, X \in \R^{n \times n}$, $\Det(A X B) \neq 0$]
\item $\Det[A X B (C^{-T} X^T C)^{-T}]$\hfill[$A, B, C, X \in \R^{n \times n}$, $\Det(C) \neq 0$, $\Det(C^{-T} X^T C) \neq 0$]
\item $\Tr[(2 I_n + a a^T)^{-1} (u v^T + v u^T)]$\hfill[$a, u, v \in \R^n$]
\item $\| u v^T - A \|_F^2 - \| A \|_F^2$\hfill[$u \in \R^m$, $v \in \R^n$, $A \in \R^{m \times n}$]
\end{enumerate}

\item Перепишите каждое из следующих выражений эквивалентным образом, используя указанные в скобках векторы/матрицы и следующие три операции: матричное произведение, транспонирование и конструирование диагональной матрицы $\Diag\{x\}$ по заданному вектору $x$. Полученное выражение не должно содержать никаких индексов.
\begin{enumerate}
\item $\sum_{i=1}^n x_i^2$\hfill[$x \in \R^n$]
\item $\sum_{i=1}^m \sum_{j=1}^n a_{ij} x_i y_j$\hfill[$A \in \R^{m \times n}$, $x \in \R^m$, $y \in \R^n$]
\item $\sum_{i=1}^n c_i a_i$\hfill[$c \in \R^n$, $A := [a_1, \dots, a_n] \in \R^{m \times n}$]
\item $\sum_{i=1}^k  u_i v_i^T$\hfill[$U := [u_1, \dots, u_k] \in \R^{m \times k}$, $V := [v_1, \dots, v_k] \in \R^{n \times k}$]
\item $B := (c_i a_{ij})_{i,j=1}^{m,n}$\hfill[$A := \R^{m \times n}$, $c \in \R^m$]
\item $B := (c_j a_{ij})_{i,j=1}^{m,n}$\hfill[$A := \R^{m \times n}$, $c \in \R^n$]
\item $\sum_{i=1}^k \sigma_i u_i v_i^T$\hfill[$\sigma \in \R^k$, $U := [u_1, \dots, u_k] \in \R^{m \times k}$, $V := [v_1, \dots, v_k] \in \R^{n \times k}$]
\item $\sum_{i=1}^k \sum_{j=1}^s \sigma_{ij} u_i v_j^T$\hfill[$\Sigma \in \R^{k \times s}$, $U := [u_1, \dots, u_k] \in \R^{m \times k}$, $V := [v_1, \dots, v_s] \in \R^{n \times s}$]
\end{enumerate}

\item Для каждого из следующих утверждений либо докажите его истинность (для произвольных значений соответствующих переменных), либо приведите пример, демонстрирующий его ложность. Если утверждение является некорректно сформированным, объясните, какие конкретно операции в нем являются недопустимыми.
\begin{enumerate}
	\item $xAx^T = x^TAx$\hfill[$x \in \R^n$, $A \in \R^{n \times n}$]
	\item $x^TAy + y^TAx = 2x^TAy$\hfill[$x, y \in \R^n$, $A \in \R^{n \times n}$]
	\item $\Det(A) \Tr(B) = \Tr(\Det(A) B)$\hfill[$A, B \in \R^{n \times n}$]
	\item $a^T x = \beta \quad \Rightarrow \quad x = (a^T)^{-1} \beta$\hfill[$a, x \in \R^n$, $\beta \in \R$]
	\item $(a a^T)x = b \quad \Rightarrow \quad x = (aa^T)^{-1} b$\hfill[$a, b, x \in \R^n$]
	\item $(I_n + a a^T)x = b \quad \Rightarrow \quad x = (I_n + aa^T)^{-1} b$\hfill[$a, b, x \in \R^n$]
	\item $A x = b \quad \Rightarrow \quad x = A^{-1} b$\hfill[$A \in \R^{m \times n}$, $b \in \R^m$, $x \in \R^n$]
	\item $u u^T \in \S^n_+$\hfill[$u \in \R^n$]
	\item $AA^T \in \S^m_+$\hfill[$A \in \R^{m \times n}$]
	\item $\Rank(A A^T) = 1$\hfill[$A \in \R^{m \times n}$]
	\item $A \in \S^n_+ \quad \Rightarrow \quad B A B^T \in \S^m_+$\hfill[$B \in \R^{m \times n}$]
	\item $A \succ 0 \quad \Leftrightarrow \quad \Det(A) > 0$\hfill[$A \in \S^n$]
\end{enumerate}

\item Пусть $A \in \S^n$. Докажите эквивалентность следующих утверждений:
\begin{enumerate}
\item (Положительная полуопределенность) $\Forall x \in \R^n: \quad x^T A x \geq 0$.
\item (Неотрицательность собственных значений): $\lambda(A) \geq 0$.
\item (Существование прямоугольного корня) $\exists D \in \R^{m \times n}: \quad A = D^T D$.
\item (Существование квадратного корня) $\exists B \in \S^n_+: \quad A = B^2$.
\end{enumerate}
(Подсказка: Используйте спектральное разложение.)

\item Для каждого из следующих уравнений найдите множество его всевозможных решений:
\begin{enumerate}
\item $a^T x = 1$\hfill[$a, x \in \R^n$]
\item $(x x^T)a = b$\hfill[$a, b, x \in \R^n$]
\item $(I_n + a a^T)x = b$\hfill[$a, b, x \in \R^n$]
\item $\left[
\begin{array}{cc}
Q & 2I \\
I & Q^T
\end{array}
\right]
\left[
\begin{array}{c}
x_1\\
x_2
\end{array}
\right]
=
\left[
\begin{array}{c}
b_1\\
b_2
\end{array}
\right]
$\hfill[$Q \in \R^{n \times n}$, $Q Q^T = I_n$, $x_1, x_2, b_1, b_2 \in \R^n$]
\end{enumerate}

\end{enumerate}

\end{document}
