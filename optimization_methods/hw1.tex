\documentclass{article}

\usepackage[T2A]{fontenc}
\usepackage[utf8x]{inputenc}
\usepackage[english,russian]{babel}

% Wide page
\usepackage{fullpage}

% Maths
\usepackage{mathtools}
\mathtoolsset{showonlyrefs} % turn off numbering of unreferenced equations
\usepackage{amsmath,amssymb,amsthm}

% Advanced enumerate
\usepackage{enumitem}

% Multiple columns
\usepackage{multicol}
\setlength\columnsep{-3em} % default columnsep for all pages

% Tables
\usepackage{diagbox} % Diagonal line in table header
\renewcommand{\arraystretch}{1.2} % More space between table rows

% Redefine some useful commands
\renewcommand{\epsilon}{\varepsilon} % nice-looking epsilon

% My operators
\newcommand{\R}{\mathbb{R}}
\renewcommand{\S}{\mathbb{S}}
\DeclareMathOperator{\Exists}{\exists}
\DeclareMathOperator{\Forall}{\forall}
\newcommand{\Tr}{\mathrm{Tr}}
\newcommand{\Det}{\mathrm{Det}}
\newcommand{\Diag}{\mathrm{Diag}}
\newcommand{\Rank}{\mathrm{Rank}}

\title{\vspace{-2em}Методы оптимизации. Домашняя работа 1.}
\author{Царькова Анастасия}
\date{}

\begin{document}

\maketitle

\section{Задача 1.}

Построим двудольный граф, в котором в правой доле $A$ вершины 1,...,13, а в левой $B$ доле 13 вершин $v_1,...,v_{13}$ отвечающие за стопки карт($v_i = {x_i, x_{i+1}, x_{i+2}, x_{i+3}}$)

В полученном двудольном графе ребрами соединены вершины $i \in A$ и $v_j \in B$ если $i \in v_j = {x_j, x_{j+1}, x_{j+2}, x_{j:eЖу
+3}}$.
Мы получили множество двудольных графов по 13 вершин в каждой доле и у каждой вершины 4 ребра.

Покажем, что по теореме Холла любой граф из построенного ранее множества имеет совершенное паросочетание.

Возьмем подмножества $A'$ и $B'$, количество различных достоинств которые лежат в $B'$ меньше либо равно $4|B'|$ и больше либо равно $|B'|$, что соответствует условию существования совершенного паросочетания в двудольном графе -- множество $B'$ переходит(по ребрам) в множество $A'$ которое больше либо равно по размеру ($|A'| \geq |B'|$).


Получили, что при любом разбиении карт на кучки, мы строим вышеупомянутый граф, находим паросочетание, и берем из каждой кучки ту карту которая равна по достоинству с значению вершины в которую из стопки входит ребро.

\section{Задача 2}

Индуктивный шаг:

Рассмотрим вершину $b$ такую, что $b \in B'$ и $b \notin M_A$ (если такая вершина существует). Тогда, возьмем ребро $e_b \in M_B$ такое между вершинами $b\in B'$ и $a \in A'$ и удалим его из нашего рассмотрения. Такая операция не нарушит паросочетания $M_A$ и $M_B$.

Получается что если существует такая вершина $b$, то можно свести задачу к меньшей.

Аналогично если существует вершина $a$, такая что $a \in A'$ и $a \notin M_B$, то так же задачу можно свести к меньшей.

База индукции:


Осталось разобрать случай, когда не существует вышеописанных вершин в множестве $A'$ и $B'$ -- значит все вершины из $A'$ покрыты ребрами из $M_B$ и вершины из $B'$ покрыты $M_A$.

Поскольку все вершины из $B'$ лежат в $M_A$, то $|A'| \geq |B'|$, аналогично для $A$ -- $|A'| \leq |B'|$, следовательно $|A'| = |B'|$

Получили, что паросочетание $M_A$ искомое для множеств $A'$ и  $B'$

Теперь, что бы получить паросочетание для исходной задачи, надо раскрутить индукцию в обратную сторону (добавлять удаленные ребра обратно).


\section{Задача 3}

Сведем задачу к $Ax = b$, $x \geq 0$, $c^Tx \rightarrow min$


$$
A =
\left(
\begin{array}{c|ccc}
			& 	-1 	& 	0\ ...\ 0	&	...\ 0\\
		M	&	0  	&	 -1	&	...\ 0\\
			& 	0	&	0\ ...\ 0	&	-1
\end{array}\right)
$$


Где $M$ -- матрица унимодулятности, а единичная матрица и имеет размер $n$x$n$ где n -- количество ребер.

Вектор  $x$ ищется таким, что бы имел вид $x = (x_1, x_2, ... x_m, y_1, ..., y_n)$ где $m$ -- количество вершин и $x_i$ и $y_i \geq 0$. Причем если $x_i$ больше нуля, то значит мы берем i-ую вершину в паросочетание(очевидно что если мы будем решать задачу минимизации то $x_i \in {0, 1}$), а $y_i$ просто какие то коэффициенты.

Вектор $b = (1, ..., 1)$.

Вектор $c = (1, ..., 1, 0,..., 0)$

Понятно, что мы получили задачу в которой минимизируем сумму "первых" ($x_i$) элементов вектора x.

Покажем, что матрица $A$ тотально унимодулярная.

Очевидно, что матрица $M$ тотально унимодулярна. Это так, потому что, поскольку граф двудольный, то значит матрицу M можно разделить на две части вдоль стороны отвечающей ребра. А так же в столбце отвечающем за ребро может стоять только две единицы, причем одна в первой половине, а вторая во второй. Получили что по лемме(не помню называется ли как нибудь она) данная матрица является тотально унимодулярной.

Теперь покажем, что добавление сбоку единичной матрицы не изменяет унимодулярности.
Если мы взяли какой то минор и в него попал столбец из этой единичной матрицы то:

1. если в столбце есть -1, то просто разложим определитель по этому с стоблцу и так сведем к задачу к меньшему размеру, причем от такого перехода не определитель изменяет только знак, значит задача остается эквивалентной.

2. если там нем -1 то весь столбец равен 0 -- значит определитель равен нулю -- что является признаком тотальной унимодулярности.

Получили что матрица A тотально унимодулярна.

\section{Задача 4}

Так как ранг матрицы A меньше n, то строки данной матрицы линейно зависимы, следовательно для данной матрицы существует такой вектора $d\ :\ Ad = 0,\ d \neq 0$.

Получили что для любого элемента из множества P есть такое $d$, что $A(v+d) = Av + Ad = Av + 0 = Av \in P$ -- следовательно нет вершин у данного полиэдра.

\end{document}
