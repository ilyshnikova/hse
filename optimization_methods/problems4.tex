\documentclass{article}
\usepackage{mynotestyle}

\noteid{Домашнее задание 4}
\notetitle{Условная оптимизация.}
\notedate{Срок сдачи: 22 марта (среда) 2017, 23:59, после срока не принимается.}

\begin{document}

\begin{mdframed}
\begin{center}
Царькова Анастасия
\end{center}
\end{mdframed}
либо сдать 21 марта на семинаре в письменном виде.

\begin{enumerate}[label=\textbf{\arabic*}, leftmargin=0em]

\item Для каждой из следующих задач найдите оптимальное значение и множество оптимальных решений:
\begin{enumerate}

\item (Линейное программирование с одним ограничением)
\begin{equation}
\min_{x \in \R^n} \{ c^T x : a^T x \leq \beta \},
\end{equation}
где $a, c \in \R^n$, $\beta \in \R$, $a \neq 0$, $c \neq 0$.


Решение:

$$
L(x, \lambda) = c^T x + \lambda (a^T x - \beta)
$$

$$
L(x, \lambda) = c^T x + \lambda (a^T x - \beta)
$$

$$
\begin{cases}
\nabla L = \nabla (c^T x + \lambda (a^T x - \beta)) &\\
\lambda (a^Tx - \beta) = 0
\end{cases}
$$

Так как $a, c \neq 0$, следовательно $\lambda \neq 0$
$$
\begin{cases}
- c^T =  \lambda a^T&\\
a^Tx = \beta
\end{cases}
$$

Получили что решение x -- гиперплоскость проходящая через точки $(a_1 / \beta, 0, ..., 0), (0, ..., 0, a_n / \beta)$, при условии что вектора $a$ и $c$ пропорциональны

\item (Линейная функция на стандартном симплексе)
\begin{equation}
\min_{x \in \R^n} \left\{ c^T x : x \succeq 0, \, \sum_{i=1}^n x_i = 1 \right\},
\end{equation}
где $c \in \R^n$.


Решение

Так как $c_i < min(c)$ тогда, поскольку $\sum_{i=1}^n x_i = 1$, $c^T x \geq (min(c), ..., min(c))x$ -- ответ -- x такие что $x = (0, ..., 0, 1, 0,  ..., 0)$ где 1 на том месте где у $c$ стоит минимальное значение.



\item (Линейная функция с энтропийным регуляризатором)
\begin{equation}
\min_{x \in \R^n_{++}} \left\{ c^T x + \sum_{i=1}^n x_i \ln x_i : \sum_{i=1}^n x_i = 1 \right\},
\end{equation}
где $c \in \R^n$.

Решение:

$$
e = (1,...,1)
$$

$$
L =  c^T x + \sum_{i=1}^n x_i \ln x_i + \lambda(e^Tx - 1)
$$

$$
\nabla_i = c_i + \ln x_i + 1 + \lambda = 0
$$

$$
\nabla_{\lambda} = e^Tx - 1 = 0
$$

$$
\ln x_i = - c_i - 1 - \lambda
$$

$$
x_i = exp(- c_i - 1 - \lambda)
$$

$$
\begin{cases}
x_i = exp(- c_i - 1 - \lambda) &\\
e^Tx - 1 = 0
\end{cases}
$$

Теперь $e$ это экспонента

$$
\begin{cases}
x_i = e^{-c_i} e^{- 1 - \lambda} &\\
\sum x_i = 1
\end{cases}
$$

$$
\begin{cases}
x_i = e^{-c_i} e^{- 1 - \lambda} &\\
e^{- 1 - \lambda} \sum e^{-c_i} = 1
\end{cases}
$$


$$
\begin{cases}
x_i = e^{-c_i} e^{- 1 - \lambda} &\\
e^{- 1 - \lambda} \sum e^{-c_i} = 1
\end{cases}
$$


$$
x_i = \frac{e^{c_i}}{\sum e^{-c_i}}
$$

Множество значений

$$
f(x^*) = \ln \sum e^{-c_i}
$$


\end{enumerate}
\item Для каждой из следующих задач оптимизации: 1) Построить двойственную задачу. 2) Выписать явные формулы, позволяющие по решению двойственной задачи восстановить (вычислить) решение прямой.
\begin{enumerate}
\item (Гребневая регрессия)
\begin{equation}
\min_{x \in \R^n, \, s \in \R^m} \left\{ \frac{1}{2} \| s - b \|_2^2 + \frac{\rho}{2} \| x \|_2^2 : s = Ax \right\},
\end{equation}
где $A \in \R^{m \times n}$, $b \in \R^m$, $\rho \in \R_{++}$.
\item (SVM)
\begin{equation}
\min_{x \in \R^n, \, t \in \R^m} \left\{ \sum_{i=1}^m t_i + \frac{\rho}{2} \| x \|_2^2 : A x \succeq 1_m - t, \, t \succeq 0 \right\},
\end{equation}
где $A \in \R^{m \times n}$, $1_m := (1, \dots, 1) \in \R^m$.
\end{enumerate}
% \item (Эквивалентная формулировка задачи через надграфик) Пусть $f : Q \to \R$~--- некоторая функция. Рассмотрим следующие две задачи оптимизации:
% \begin{equation}
% (P_1) \quad \min_{x \in Q} f(x) \hspace{10em} (P_2) \quad \min_{x \in Q, \, t \in \R} \{ t : t \geq f(x) \}
% \end{equation}
% Докажите, что задачи $(P_1)$ и $(P_2)$ являются эквивалентными, в том смысле, что
% \begin{enumerate}
% \item Оптимальные значения этих задач совпадают: $P_1^* = P_2^*$, где $P_1^* := \inf_{x \in Q} f(x)$ и $P_2^* := \inf_{x \in Q, \, t \in \R} \{ t : t \geq f(x) \}$ (при этом допускается случай, что $P_1^* = -\infty$ или $P_2^* = -\infty$).
% \item Множества оптимальных решений этих задач совпадают с точностью до проекции: $\mathrm{Opt}(P_1) = \pi_1\left( \mathrm{Opt}(P_2) \right)$, где $\mathrm{Opt}(P_1) := \{ x \in Q : f(x) = P_1^* \}$, $\mathrm{Opt}(P_2) := \{ (x, t) \in Q \times \R : (t \geq f(x)) \land (t = P_2^*) \}$ и $\pi_1$~--- оператор проектирования множества пар~$(x, t)$ в множество соответствующих первых элементов~$x$ (например, $\pi_1(\{(1, 1), (2, 1), (3, 2)\}) = \{ 1, 2, 3 \}$).
% \end{enumerate}

%\item Используя прием из предыдущей задачи (или аналогичный ему), свести эквивалентным образом следующие негладкие безусловные задачи к гладким условным:
\newpage

%\item Используя прием из предыдущей задачи (или аналогичный ему), свести эквивалентным образом следующие негладкие безусловные задачи к гладким условным:
\newpage

%---------------------------------------------------
\item Свести эквивалентным образом следующие негладкие безусловные задачи к гладким условным:
\begin{enumerate}
\item (Максимум из конечного числа гладких функций)
\begin{equation}
\min_{x \in \R^n} \max\{f_1(x), \dots, f_m(x) \},
\end{equation}
где $f_i : \R^n \to \R$~--- заданные гладкие функции.

Решение:

Рассмотрим любое $t$ и любое $x$, для которых $t > f_i(x)$. Для них справедливо, что

$$
t > max\{f_1(x),\ldots,f_m(x)\} \geq \min_{x \in \mathbf{R}^n} max\{f_1(x),\ldots,f_m(x)\}
$$

Поскольку это неравенство верно для всех $t$, то оно верно и для минимума по $x, t$, то есть

$$\min_{x,t, t > f_i(x)} t \geq \min_{x \in \mathbf{R}^n} max\{f_1(x),\ldots,f_m(x)\}$$

С другой стороны, для любого $x, \varepsilon > 0$ рассмотрим

$$
t = max{f_i(x)} + \varepsilon
$$

Это выражение больше,чем все $f_i(x)$. Поэтому, оно больше, чем $\min_{x,t, t > f_i(x)} t$. Но поскольку это происходит для любого сколь угодно малого $\varepsilon$, то

$$
max{f_i(x)} \geq \min_{x,t, t > f_i(x)} t
$$

Поскольку это происходит для всех $x$, то

$$\min_{x \in \mathbf{R}^n} max\{f_1(x),\ldots,f_m(x)\} \geq \min_{x,t, t > f_i(x)} t $$

В итоге, получаем, что требуемые выражения равны.

\item (Наилучшее решение линейной системы в $\ell_{\infty}$-норме)
\begin{equation}
\min_{x \in \R^n} \| A x - b \|_{\infty},
\end{equation}
где $A \in \R^{m \times n}$, $b \in \R^m$. Для произвольного вектора $y \in \R^m$: $\| y \|_{\infty} := \max_{i=1}^m |y_i|$.
\item (Наилучшее решение линейной системы в $\ell_1$-норме)
\begin{equation}
\min_{x \in \R^n} \| A x - b \|_1,
\end{equation}
где $A \in \R^{m \times n}$, $b \in \R^m$. Для произвольного вектора $y \in \R^m$: $\| y \|_1 := \sum_{i=1}^m |y_i|$.
\item (Задача LASSO)
\begin{equation}
\min_{x \in \R^n} \left\{ \frac{1}{2} \| A x - b \|_2^2 + \rho \| x \|_1 \right\},
\end{equation}
где $A \in \R^{m \times n}$, $b \in \R^m$, $\rho \in \R_{++}$.

%\emph{Указание:} Воспользуйтесь тем, что $|x_i| = \max\{x_i, -x_i\}$ и введите по одной новой переменной $t_i$ для каждого максимума.
\end{enumerate}

% (Бойд, 4.21)
\item Для каждой из следующих квадратичных задач (QCQP) найдите аналитическое решение:
\begin{enumerate}
\item (Минимизация линейной формы на эллипсоиде)
\begin{equation}
\min_{x \in \R^n} \{ c^T x : x^T A x \leq 1 \},
\end{equation}
где $c \in \R^n \setminus \{0\}$ и $A \in \S^n_{++}$.

Решение:

Существует матрица $D$ такая, что

$$
A = D^T D
$$
Сделаем замену

$$
y = Dx
$$
Матрица $D$ -- обратима, поскольку имеет ненулевой определитель, поскольку матрица $A$ строго-положительно определена. Поэтому

$$
x = D^{-1}y
$$
В итоге, получается, что нам нужно найти

$$
\min_{x \in \mathbf{R}^n} \left\{c^T D^{-1}y : y^T y \leq 1\right\}
$$
Иными словами, найти миниум выражения
$$
k_1 y_1 + \ldots + k_n y_n
$$
при условии

$$
y_1^2 + \ldots + y_n^2 \leq 1
$$
Выписав производные Лагранжиана, получаем условия

$$
k_1 = 2\lambda y_1
$$
то есть оптимальные значeния $y_i$ пропорциональны $k_i$. Из условия на сумму квадратов, полуачем такой ответ

$$
y_i = \displaystyle\frac{-k_i}{\sqrt{k_1^2 + \ldots + k_n^2}}
$$
где $k_i$ -- коэффициенты вектора $c^TD^{-1}$.
\item (Минимизация квадратичной формы на эллипсоиде)
\begin{equation}
\min_{x \in \R^n} \{ x^T B x : x^T A x \leq 1 \},
\end{equation}
где $A \in \S^n_{++}$, $B \in \S^n_+$.

Очевидно, что

$$
x^T B x \geq 0
$$
Кроме того, ноль достигается при $x = 0$, поэтому $x = 0$ и есть ответ.

\end{enumerate}

\item Для каждого из следующих множеств $Q \subseteq \R^n$ найти евклидову проекцию заданной точки $v \in \R^n$ на множество $Q$ (т.~е. найти $\Proj_Q(v) := \argmin_{x \in Q} \| x - v \|_2^2$):
\begin{enumerate}
\item (Короб) $Q = \{ x \in \R^n : x_i \in [l_i, r_i], \ i = 1, \dots, n \}$, где $-\infty \leq l_i \leq r_i \leq +\infty$. (\emph{Замечание:} Допускается, что $l_i = -\infty$ и/или $r_i = +\infty$, т.~е. короб может быть неограниченным вдоль некоторых направлений.)

Рассмотрим произвольную точку $v$. Надо найти для нее самую ближайщую в коробе, для этого надо минимизировать сумму

$$
|x_1 - v_1|^2 + \ldots + |x_n - v_n|^2
$$
Eсли

$$
v_i \in [l_i, r_i]
$$
то возмем

$$
x_i = v_i
$$
Если же

$$
v_i > r_i
$$
то возьмем
$$
x_i = r_i
$$
И, наконец, если

$$
v_i < l_i
$$
возьмем

$$
x_i = l_i
$$
Понятно, что таким образом мы минимизируем каждое слагамое $|x_i - v_i|$ по-отдельности. И, плюс к этому, построенная точка действительно лежит в коробе, поэтому она и минимизирует всю сумму квадратов модулей

%\item (Единичный $L_2$-шар) $Q = B_2(0, 1) = \{ x \in \R^n : \| x \|_2 \leq 1 \}$.
\item (Аффинное многообразие) $Q = \{ x \in \R^n : A x = b \}$, где $A \in \R^{m \times n}, \, b \in \R^m, \, \Rank(A) = m$.

%Найдем сначала $m$ линейно-независимых столбцов матрицы $A$. Ограничение матрицы на эти столбцы является квадратной матрицей с ненулевым определителем. Возьмем к этой матрице обратную матрицу $X$.

%Тогда, нетрудно понять, что при умножении всей матрицы $X$  на матрицу $A$, получится матрица, у которой ограничения на эти столбцы равно единичной матрице, а уравнение примет вид

%$$
%XAx = Xb
%$$
%Теперь получается, что решение принимает следующий вид: на координаты, номера которых не содержатся среди номеров выделенных столбков, мы можем поставить любые числа в вектор $x$. А затем, выбрав взаимно-однозначно коррдинаты на номерах поцизий, соответствующих выделенным столбцам, мы можем добиться, чтобы уравнение выполнялось. То есть, решение имеет вид


\item (Полупространство) $Q = \{ x \in \R^n : a^T x \leq \beta \}$, где $a \in \R^n, \, \beta \in \R, \, a \neq 0$.

Если $v$ лежит в полуплоскости,то $x = v$. Если нет, то надо выбрать ортогональную проекцию, то есть

$$
x = v - \lambda a
$$
Найдем $\lambda$:

$$
a^T v - a^T a \lambda = b
$$
$$
\lambda = \displaystyle\frac{a^Tv - b}{a^Ta}
$$

В итоге, ответ
$$
x = v - \displaystyle\frac{a^Tv - b}{a^Ta} a
$$

\end{enumerate}
Воспользуйтесь полученными выше результатами и выпишите ответ для следующих случаев:
\begin{itemize}
\item (Неотрицательный ортант) $Q = \R^n_+ = \{ x \in \R^n : x_i \geq 0, \ i = 1, \dots, n \}$.
Нужно взять такой вектор $x$, что

$$
x_i = min(0, v_i)
$$
\item (Единичный $L_{\infty}$-шар) $Q = B_{\infty}(0, 1) = \{ x \in \R^n : \| x \|_{\infty} \leq 1 \}$.
Нужно взять
$$
x_i = sgn(v_i) \cdot max(1, |v_i|)
$$
\item (Гиперплоскость) $Q = \{ x \in \R^n : a^T x = \beta \}$, где $a \in \R^n, \, \beta \in \R, \, a \neq 0$.
$$
x = v - \displaystyle\frac{a^Tv - b}{a^Ta} a
$$

\end{itemize}

%---------------------------------------------------

%\warning{Нужно что-нибудь еще на матрицы и на двойственные задачи}

\section*{Бонусная часть (6 баллов)}

%\item (Бойд, 4.22)
\item Рассмотрим QCQP:
\begin{equation}
\min_{x \in \R^n} \left\{ \frac{1}{2} x^T A x - b^T x : \| x \|_2 \leq 1 \right\},
\end{equation}
где $A \in \S^n_{++}$ и $b \in \R^n$. Докажите, что оптимальное решение в этой задаче равно $(A + \lambda I_n)^{-1} b$, где $\lambda := \max\{ 0, \bar{\lambda} \}$ и $\bar{\lambda}$~--- это наибольшее из решений нелинейного уравнения
\begin{equation}
b^T (A + \lambda I_n)^{-2} b = 1.
\end{equation}

Решение:

x -- отптимально если

1. $x^Tx < 1\ --\ Ax + b = 0$ тогда $Ax = b$ и если $\|A^{-1}b\|_2 \leq 1$


2. $x^Tx = 1\ --\  Ax + b = -\lambda x\ \forall \lambda \geq 0$. Из условий $\|x\|_2 = 1$, и пусть $\lambda_i$ -- собственные числа матрицы A -- тогда определим $f(\lambda) = \|(A + \lambda)^{-1}b\|_2^2 = \sum_{i = 1}^{n} \frac{b_i^2}{(\lambda + \lambda_i)^2}$ тогда для этой функции $f(0) = \|A^{-1}b\|_2^2 > 1$. Полученная функция монотонно убыает при стрмлении $\lambda$ к нулю -- поэтому уравнение $Ax + b = -\lambda x$ эквивалентное $f(\lambda) = 0$ имеет только одно единственное рещение а значит $x = -(A + - \lambda I)^{-1}q$ -- оптимальное решение




\item Рассмотрим задачу поиска евклидовой проекции заданной точки $v \in \R^n$ на стандартный симплекс:
\begin{equation}
\Proj_{\Delta_n}(v) := \argmin_{x \in \R^n} \left\{ \| x - v \|_2 : x \succeq 0, \, \sum_{i=1}^n x_i = 1 \right\}.
\end{equation}
Докажите, что $\Proj_{\Delta_n}(v) = (v - \nu 1_n)_+$, где $\nu \in \R$~--- корень нелинейного уравнения
\begin{equation}\label{eq:proj_simplex_eq}
1_n^T (v - \nu 1_n)_+ = 1.
\end{equation}
Здесь $1_n := (1, \dots, 1) \in \R^n$, а $(u)_+$ обозначает поэлементную положительную срезку $(u_i)_+ := \max\{0, u_i\}$.
Нарисуйте схематичный график левой части уравнения~\eqref{eq:proj_simplex_eq} как функции от $\nu$.

\emph{Подсказка}. Удобно рассмотреть упорядоченные компоненты $v_{[1]} \geq \dots \geq v_{[n]}$.

\item Пусть $\Sigma, \Sigma_0 \in \S^n_{++}$. Обозначим через $D(\Sigma; \Sigma_0)$ \emph{дивергенцию Кульбака--Лейблера} между нормальными распределениями $\mathcal{N}(0, \Sigma)$ и $\mathcal{N}(0, \Sigma_0)$:
\begin{equation}
D(\Sigma; \Sigma_0) = \frac{1}{2} ( \Tr(\Sigma_0^{-1} \Sigma) - \ln \Det(\Sigma_0^{-1} \Sigma) - n ).
\end{equation}
Пусть $H \in \S^n_{++}$. Пусть также $y, s \in \R^n$, причем $y^T s > 0$. Рассмотрим задачу поиска матрицы $X \in \S^n_{++}$, минимизирующую дивергенцию $D(X^{-1}; H^{-1})$ при условии $X y = s$:
\begin{equation}
\min_{X \in \S^n_{++}} \{ D(X^{-1}; H^{-1}) : X y = s \}.
\end{equation}
Решите эту задачу и убедитесь, что ее решение выражается по формуле обновления обратной матрицы в схеме BFGS:
\begin{equation}
X = (I_n - \rho s y^T) H (I_n - \rho y s^T) + \rho s s^T,
\end{equation}
где $\rho := 1/(y^T s)$.

\end{enumerate}

\end{document}
