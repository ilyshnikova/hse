\documentclass{article}

\usepackage[T2A]{fontenc}
\usepackage[utf8x]{inputenc}
\usepackage[english,russian]{babel}

% Wide page
\usepackage{fullpage}

% Maths
\usepackage{mathtools}
\mathtoolsset{showonlyrefs} % turn off numbering of unreferenced equations
\usepackage{amsmath,amssymb,amsthm}

% Advanced enumerate
\usepackage{enumitem}

% Nice frameboxes
\usepackage{mdframed}

% Advanced verbatim
\usepackage{fancyvrb}

% Hyperlinks
\usepackage{hyperref}

% Multiple columns
\usepackage{multicol}
\setlength\columnsep{-3em} % default columnsep for all pages

% Tables
\usepackage{diagbox} % Diagonal line in table header
\renewcommand{\arraystretch}{1.2} % More space between table rows

% Redefine some useful commands
\renewcommand{\epsilon}{\varepsilon} % nice-looking epsilon

% My operators
\newcommand{\R}{\mathbb{R}}
\renewcommand{\S}{\mathbb{S}}
\DeclareMathOperator{\Exists}{\exists}
\DeclareMathOperator{\Forall}{\forall}
\DeclareMathOperator*{\Tr}{Tr}
\DeclareMathOperator*{\Det}{Det}
\DeclareMathOperator*{\Diag}{Diag}
\DeclareMathOperator*{\Rank}{Rank}
\DeclareMathOperator*{\Dom}{Dom}

\title{\vspace{-2em}Методы оптимизации (ФКН ВШЭ, 2017). Домашняя работа 3.\\Тема: Выпуклые множества и функции.}
\date{}

\begin{document}

\maketitle

\begin{mdframed}
\begin{center}
Домашнее задание 3, Царькова Анастасия.
\end{center}
\end{mdframed}

\begin{enumerate}[label=\textbf{\arabic*}, leftmargin=0em]

\section*{Обязательная часть (10 баллов)}

\item Покажите, что единичная сфера $\{ x \in \mathbb{R}^n \, | \, \|x\| = 1 \}$ не является выпуклым множеством. Здесь $\| \cdot \|$ --- произвольная норма.

Решение:

Рассмотрим пример из $\mathbb{R}^2$. Возьмем на сфере две противоположных точки, например $(0, 1)$ и $(0, -1)$. Возьмем точку посередине -- $(0,1)\cdot0.5 + (0,-1)\cdot0,5 = (0,0)$. Проверим лежит ли полученная точка на сфере -- $\|(0,0)\| \neq 1$ -- нет --> данное множество не выпуклое.

\item Какие из следующих множеств являются выпуклыми? Ответ обосновать.
\begin{enumerate}

\begin{multicols}{2}
    \item $\{ x \in \mathbb{R}^n \, | \, \max_{i} x_i \leqslant 1\}$
    \item $\{ x \in \mathbb{R}^n \, | \, \max_{i} x_i \geqslant 1\}$

    \item $\{ x \in \mathbb{R}^n \, | \, \min_{i} x_i \leqslant 1\}$
    \item $\{ x \in \mathbb{R}^n \, | \, \min_{i} x_i \geqslant 1\}$
\end{multicols}

Решение:

(a) Рассмотрим два элемента из данного множества $a,b$. Тогда для элемента $ c_i = a_i\lambda + b_i(1 - \lambda) \leq  \lambda + (1 - \lambda) = 1$, а следовательно $\max_{i} c_i \leq 1$


(b) Рассмотрим два элемента данного множества -- $(1,0), (0,1)$ и рассмотрим среднюю точку между ними -- $(0,1)\cdot0.5 + (1,0)\cdot0,5 = (0.5,0.5)$ -- полученная точка не пренаджежит исходному множеству --> множество не выпуклое.

(c) Рассмотрим два элемента данного множетсва $(x,1),(1,x)$ и рассмотрим среднюю точку между ними -- $(x,1)\cdot0.5 + (1,x)\cdot0,5 = (0.5 + x/2,0.5 + x/2 )$ -- для $x > 1$ верно что $x/2 > 0.5$ --> $\min_{i} (0.5 + x/2,0.5 + x/2)_i \geq 1$ -- множество не выпуклое.

(d) Рассмотрим два элемента из данного множества $a,b$. Тогда для элемента $ c_i = a_i\lambda + b_i(1 - \lambda) \geq  \lambda + (1 - \lambda) = 1$, а следовательно $\max_{i} c_i \geq 1$


\end{enumerate}

\item Покажите выпуклость множества $\{ x \in \R^n \; \big| \; x^T P x \leqslant (c^T x)^2, \, c^T x \geqslant 0 \}$, где $c \in \R^n, \; P \in \S_{++}^n$.

Решение:

Рассмотрим два элемента из данного множества $a,b$. По неравенству Коши-Буняковского: $\sqrt{a^TPbb^TPa} \leq a^TPb$. Тогда рассмотрим точку между a и b:

$$
(c^T(a\lambda + (1-\lambda)b))^2 = (c^Ta)\lambda^2 + 2(1-\lambda)\lambda c^Tac^Tb + (c^Tb)^2(1 - \lambda)^2 \geq
$$
$$
\geq \lambda^2a^TPa + 2(1-\lambda)\lambda \sqrt{a^TPa b^TPb} + (1 - \lambda)^2 b^TPb \geq \lambda^2a^TPa + 2(1-\lambda)\lambda a^TPb + (1 - \lambda)^2 b^TPa = ((1-\lambda)b + \lambda a)^T P ((1-\lambda)b + \lambda a)
$$

Получили что линейная комбинация a и b принадлежит данному множеству --> множество является выпуклым.


\item Покажите, что следующие функции являются выпуклыми:
\begin{enumerate}
    \item $\displaystyle f(x) = \sum_{i = 1}^n w_i \ln(1 + \exp(a_i^T x)) + \frac{\mu}{2}\|x\|^2_2, \qquad \mu > 0, \; w_i > 0, \; a_i \in \mathbb{R}^n, \quad \Dom f~:=~\mathbb{R}^n$.

    \item $\displaystyle f(x) = \max_{1 \leq i \leq n} \Bigl\{ w_i \ln\bigl(1 + \exp( |x_i| )\bigr) \Bigr\}, \qquad w_i > 0, \; \quad \Dom f := \mathbb{R}^n$.

    \item $\displaystyle f(X) = \Tr( X^{-1} ), \qquad \Dom f := \mathbb{S}^n_{++}$.

    \item $\displaystyle f(x) = (a^T x - b)_+, \qquad a \in \mathbb{R}^n, \; b \in \mathbb{R}, \quad \Dom f := \mathbb{R}^n$.

    \item $\displaystyle f(x) = \ln \left( \sum_{i = 1}^n \exp( [(x_i)_+]^2 ) \right), \qquad \Dom f := \mathbb{R}^n$.
\end{enumerate}

(a) -


(b) -


(c) Найдем втору производную

$$
d(\Tr( X^{-1} )) = - \Tr( X^{-1}HX^{-1})
$$

$$
d^2(- \Tr( X^{-1}HX^{-1})) = 2Tr(X^{-1}HX^{-1}HX^{-1})
$$

$$
\|A\|_{F} = Tr(A^TA)^{1/2}
$$

$$
2Tr(X^{-1}HX^{-1}HX^{-1}) = 2Tr(X^{-1}HX^{-1/2}X^{-1/2}HX^{-1}) = \|X^{-1}HX^{-1/2}\|_{F}^2 > 0
$$

Следовательно функция является выпуклой.


(d) Рассмотрим два элемента из данного множества $a,b$. Линейная сумма данных элементов

$$
f(c) = f(x\lambda + y(1 - \lambda)) = (a^T(x\lambda + y(1 - \lambda)) - b = a^Tx\lambda + a^Ty(1 - \lambda)) - b)_+ = (a^Tx\lambda + a^Ty - a^T\lambda - b)_+ \geq
$$

$$
\geq (a^Tx\lambda - a^Ty\lambda + \lambda b - \lambda b)_+ + (a^Ty - b)_+ = f(y) + \lambda f(x) - \lambda f(y) =  \lambda f(x) + (1-\lambda) f(y)
$$

Функция выпуклая.

(Обозначение: $(t)_+ := \max\{t, 0\}$ --- положительная срезка.)

\item Пусть $f: Q \rightarrow \R$, где $Q$ --- выпуклое множество. Докажите эквивалентность следующих утверждений:
\begin{enumerate}
\item Функция $f$ является выпуклой: $f(\lambda x + (1 - \lambda) y) \leqslant \alpha f(x) + (1 - \lambda) f(y)$ для всех $x, y \in Q$, $\lambda \in (0, 1)$.
\item Надграфик $\text{Epi}(f) := \{ (x, t) \in Q \times \mathbb{R} \, | \, f(x) \leqslant t \}$ является выпуклым множеством.
\end{enumerate}


Решение:


Покажем a -> b.

Пусть $f(a) \geq t_1$ и $f(b) \geq t_2$.

Рассмотрим линейную комбинацию -- $f(\lambda a + (1 - \lambda) b) \leq \lambda f(a) + (1- \lambda) b \leq \lambda t_1 + (1 - \lambda) t_2$ следовательно надргафиг выпуклое мн.


Покажем b -> a.

Пусть $f(a) = t_1$ и $f(b) = t_2$.

$f(\lambda x + (1 - \lambda) y) $ (по условию) $ \leq \lambda t_1 + (1 - \lambda) t_2 < \lambda f(a) + (1 - \lambda) f(b)$ — f выпуклая






\bigskip

\section*{Бонусная часть (6 баллов)}

\item Для каждой из следующих функций определите, является ли она выпуклой? Вогнутой?
\begin{enumerate}
\item $\displaystyle f(x) = \left( \sum_{i=1}^n x_i^p \right)^{\frac{1}{p}}, \quad p < 1, \ p \neq 0, \qquad \Dom f := \R_{++}^n$.

\item (Минимальное сингулярное число) $f(X) = \sigma_{\min}(X), \qquad \Dom f := \R^{m \times n}$.

\item (Среднее геометрическое компонент) $\displaystyle f(x) = \left(\prod_{i=1}^n x_i \right)^{1/n}, \qquad \Dom f~:=~\R_{+}^n$.

\item (Среднее геометрическое собственных значений) $\displaystyle f(X) = \left( \prod_{i=1}^n \lambda_i(X) \right)^{1/n}, \qquad \Dom f := \S^n_{+}$.

\item (Сумма $k$ старших компонент) $\displaystyle f(x) = \sum_{i=1}^k x_{[i]}, \qquad 1 \leqslant k \leqslant n, \qquad \Dom f := \R^n$.\\
(Здесь $x_{[i]}$ обозначает $i$-ую компоненту отсортированного по убыванию вектора $x$.)
\end{enumerate}

\item Рассмотрим функцию двух аргументов:
\begin{equation}
f(x) =
\begin{cases}
\frac{x_2^2}{x_1}, \quad &x_1 > 0, \\
0, \quad &x_1 = x_2 = 0,
\end{cases} \qquad \Dom f := \R_+ \times \R.
\end{equation}
Покажите, что надграфик $\text{Epi}(f) := \{ (x, t) \in \Dom f \times \R \, | \, f(x) \leqslant t  \}$ является выпуклым множеством и, тем самым, функция $f$ является выпуклой (хоть и разрывной).

\item Пусть $\mathcal{F} \subset C^1(\R^n)$ --- максимальное по включению подмножество непрерывно-дифференцируемых функций на $\R^n$, удовлетворяющее следующим трем требованиям.
\begin{itemize}
\item Для произвольной функции $f \in \mathcal{F}$ условие оптимальности первого порядка в некоторой точке является \emph{достаточным} для того, чтобы эта точка была глобальным минимумом функции:
\begin{equation}
\Bigl(\nabla f(x_0) = 0, \ x_0 \in \R^n \Bigr) \quad \Rightarrow \quad \Bigl( f(x) \geq f(x_0) \; \text{ для всех } x \in \R^n \Bigr),
\end{equation}
\item Класс $\mathcal{F}$ замкнут относительно неотрицательных линейных комбинаций:
\begin{equation}
\Bigl(f_1, f_2 \in \mathcal{F}, \; \; \alpha, \beta \geqslant 0\Bigr) \quad \Rightarrow \quad \Bigl(\alpha f_1 + \beta f_2 \in \mathcal{F}\Bigl),
\end{equation}
\item Класс $\mathcal{F}$ содержит все аффинные функции:
\begin{equation}
\Bigl(f(x) = a^T x + b, \quad a \in \R^n, \; b \in \R \Bigr) \quad \Rightarrow \quad \Bigl(f \in \mathcal{F}\Bigl).
\end{equation}
\end{itemize}
Докажите, что $\mathcal{F}$ состоит в точности из всех непрерывно-дифференцируемых выпуклых функций.

\end{enumerate}

\end{document}
