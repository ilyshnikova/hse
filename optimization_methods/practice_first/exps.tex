% !TEX encoding = UTF-8 Unicode
\documentclass{article}
%\documentclass[12pt,reqno]{amsart}
\usepackage[russian]{babel}
\usepackage[utf8]{inputenc}
%\usepackage[dvips]{graphicx,graphics}
\usepackage{graphicx}
\usepackage{euscript}
\usepackage{graphics}
%\usepackage{russcorr}
\usepackage[active]{srcltx} % SRC Specials: DVI [Inverse] Search
\usepackage{amssymb,amsmath,amsthm,amsfonts}
\usepackage{amsopn}
\newtheorem{cor}{Следствие}
\newtheorem{lem}{Лемма}
\newtheorem{thm}{Теорема}
\newtheorem{prop}{Предложение}
\newtheorem*{thm_pres}{Теорема}
\theoremstyle{definition}
\newtheorem{defn}{Определение}
\newtheorem{defneq}{Эквивалентное определение}
\theoremstyle{remark}
\newtheorem*{rem}{Замечание}
\newtheorem*{deff}{Обозначение}
\usepackage{verbatim}
\usepackage{listings}
\usepackage{hyperref}
\usepackage{color}

\definecolor{dkgreen}{rgb}{0,0.5,0}
\definecolor{gray}{rgb}{0.5,0.5,0.5}
\definecolor{mauve}{rgb}{0.58,0,0.52}

\lstset{frame=tb,
  language=Java,
  aboveskip=3mm,
  belowskip=3mm,
  showstringspaces=false,
  columns=flexible,
  basicstyle={\small\ttfamily},
  numbers=none,
  numberstyle=\tiny\color{gray},
  keywordstyle=\color{blue},
  commentstyle=\color{dkgreen},
  stringstyle=\color{mauve},
  breaklines=true,
  breakatwhitespace=true,
  tabsize=3
}

\newcommand{\sug}[1]{\rule[-2mm]{0.4pt}{5mm}_{\,{#1}}}
\newcommand{\gen} {$GE^+_n(\mathbf{R}[x])\ $}
\newcommand{\genn} {$GE^+_2(\mathbf{R}[x])\ $}
\newcommand{\gn} {$G_n(\mathbf{R})\ $}
\newcommand{\gln} {$GL_n(\mathbf{R}[x])\ $}
\newcommand{\p} {\mathbf{P}}
\newcommand{\peq} {$\mathcal{P}$}
\newcommand{\po} {$\mathcal{P}_0$}
\newcommand{\ff} {$\mathbf{R}\ $}
\newcommand{\fx} {\mathbf{R}[x]}
\newcommand{\fp} {\mathbf{R_+}}
\newcommand{\fxp} {\mathbf{R_+}[x]}
\newcommand{\zx} {\mathbb{Z}[x]}
\newcommand{\zxp} {\mathbb{Z_+}[x]}
\newcommand{\basering} {$\mathbf{F}$}
\newcommand{\lfrac} [2] {\displaystyle \frac{#1}{#2}}
\newcommand{\brsum} [3] {\displaystyle \sum \limits_{#1}^{#2} \left( #3\right)}
\newcommand{\lsum} [2] {\displaystyle \sum \limits_{#1}^{#2}}
\newcommand{\br} [1] {\left( #1 \right)}
\newcommand{\tab} {\mbox{             } \quad }
\usepackage{a4wide}

\usepackage{verbatim}
\usepackage{hyperref}
\hypersetup{
    colorlinks,
    citecolor=black,
    filecolor=black,
    linkcolor=blue,
    urlcolor=blue
}
\setcounter{secnumdepth}{0}


\begin{document}
\section*{Отчет по первому практическому заданию}
\section*{Царькова Анастасия}

Формулы для градиента и гессиана функции логистической регрессии:

$$
f(x) = \lfrac{1}{m} \sum_{i = 1}^m \ln(1 + \exp(-b_ia_i^Tx)) + \lfrac{\lambda}{2} \|x\|_2^2 = \lfrac{1}{m} \ln(1 + \exp(-b * Ax)) + \lfrac{\lambda}{2} x^Tx
$$

$$
\nabla f(x) = \lfrac{1}{m} A^T \left(b * \left(\lfrac{1}{1 - \exp(b * Ax)} \right)\right) + \lambda x
$$

$$
\nabla^2 f(x) = \lfrac{1}{m} A^T Diag\left(\left(1 - \left(\lfrac{1}{1 - \exp(b * Ax)} \right)\right) * \left(\lfrac{1}{1 - \exp(b * Ax)} \right)\right)A + \lambda I
$$


\section{Первый эксперимент.}

Зависимость поведения метода от обусловденности функции.

\includegraphics[width=0.5 \textwidth]{11.png}
\includegraphics[width=0.5 \textwidth]{12.png}

\includegraphics[width=0.5 \textwidth]{13.png}
\includegraphics[width=0.5 \textwidth]{14.png}

\includegraphics[width=0.5 \textwidth]{15.png}
\includegraphics[width=0.5 \textwidth]{16.png}

\includegraphics[width=0.5 \textwidth]{17.png}


Зависимость поведения метода от начальной точки

\includegraphics[width=0.5 \textwidth]{121.png}
\includegraphics[width=0.5 \textwidth]{122.png}

\includegraphics[width=0.5 \textwidth]{123.png}
\includegraphics[width=0.5 \textwidth]{124.png}

\includegraphics[width=0.5 \textwidth]{125.png}


Зависимость поведения метода от стратегии выбора шага.

\includegraphics[width=0.6 \textwidth]{131.png}
\includegraphics[width=0.6 \textwidth]{132.png}

\includegraphics[width=0.6 \textwidth]{133.png}

Зависимость числа итераций градиентного спуска от числа обусловленности и размерности пространства.

\includegraphics[width=1 \textwidth]{21.png}

Сравнение методов градиентного спуска и Ньютона на реальной задаче логистической регрессии.

w8a

\includegraphics[width=0.6 \textwidth]{31.png}
\includegraphics[width=0.6 \textwidth]{32.png}

gisette-scale

\includegraphics[width=0.6 \textwidth]{33.png}
\includegraphics[width=0.6 \textwidth]{34.png}

real-sim

\includegraphics[width=0.6 \textwidth]{35.png}
\includegraphics[width=0.6 \textwidth]{36.png}

Оптимизация вычислений в градиентном спуске.


\includegraphics[width=0.6 \textwidth]{41.png}


\includegraphics[width=0.6 \textwidth]{42.png}
\includegraphics[width=0.6 \textwidth]{43.png}

\end{document}
